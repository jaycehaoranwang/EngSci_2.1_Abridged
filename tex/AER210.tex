\documentclass[a4paper,12pt]{report}

\usepackage{amsmath,amsfonts,mathtools}
\usepackage{hyperref}

\begin{document}
\title{AER210 Abridged}
\author{Aman Bhargava}
\date{September 2019}
\maketitle

\tableofcontents

\pagebreak

\chapter{Review: Stuff to Have Memorized}
%TODO: Populate this with stuff from MAT195
\section{Trig Functions and Derivatives}
\def\arraystretch{2}%
\begin{tabular}{cc}
$ \frac{d}{dx}sin(x) = cos(x) $ & $ \frac{d}{dx}csc(x) = -csc(x)cot(x) $ \\
$ \frac{d}{dx}cos(x) = -sin(x) $ & $ \frac{d}{dx}sec(x) = sec(x)tan(x) $ \\
$ \frac{d}{dx}tan(x) = sec^2(x) $ & $ \frac{d}{dx}cot(x) = -csc^2(x) $ \\
\end{tabular}

\section{Inverse Trig Derivatives}
\def\arraystretch{2}%
\begin{tabular}{cc}
$ \frac{d}{dx}sin^{-1}(x) = \frac{1}{\sqrt(1-x^2)} $ \\
$ \frac{d}{dx}cos^{-1}(x) = \frac{-1}{\sqrt(1-x^2)} $ \\
$ \frac{d}{dx}tan^{-1}(x) = \frac{1}{1+x^2} $ \\
\end{tabular}

\section{How to complete the Square}
\begin{enumerate}
\item Put $ax^2 + bx$ in brackets and forcefully factor out the $a$
\item Add $ ( \frac{b}{2})^2$ to the inside of the brackets and subtract it from the outside (you got it)
\item Factor and be happy that you've completed the square;
\end{enumerate}

\section{Trig Angle Sums}
\begin{enumerate}
\item $sin(A+B) = sin(A)cos(B) + cos(A)sin(B)$
\item $cos(A+B) = cos(A)cos(B) - sin(A)sin(B)$
\item $sin(A-B) = sin(A)cos(B) - cos(A)sin(B)$
\item $cos(A-B) = cos(A)cos(B) + sin(A)sin(B)$
\end{enumerate}

\section{Identities Assorted}
\begin{eqnarray}
sin^2 (x) = 1/2 (1-cos(2x)) \\
cos^2 (x) = 1/2 (1+cos(2x)) \\
sinxcosx = 1/2 sin2x
\end{eqnarray}

\begin{eqnarray}
sin A cos B = 1/2 [sin(A-B) + sin(A+B)] \\
sin A sin B = 1/2 [cos(A-B) - cos(A+B)] \\
cos A cos B = 1/2 [cos(A-B) + cos(A+B)]
\end{eqnarray}

\begin{eqnarray}
\frac{d}{dx} csc(x) = -csc(x)cot(x) \\
cot^2(x) = csc^2(x) - 1 \\
\frac{d}{dx} cot(x) = -csc^2(x)
\end{eqnarray}




\chapter{In-Class Review}
\section{Vectors \& Vector Functions}

\begin{itemize}
\item Vector = magnitude + direction
\item If the origin of the vector is the origin of the coordinate system, it's a position vector.
\item Dot product: 
$\vec{a} \cdot \vec{b} = \vec{a}_1 \cdot \vec{b}_1 + ... + \vec{a}_n \cdot \vec{b}_n$
\item Cross product: $\vec{a} \times \vec{b} = det(i, j, k; \vec{a}^{T}; \vec{b}^{T})$
\item Cross product is the area of the paralellogram traced out by the two vectors.
\item Scalar triple product: $\vec{a} \cdot (\vec{b} \times \vec{c})$, produces a scalar, represents the volume of the parallelepiped formed by the three vectors.
\item To get the derivative of a vector function, simply  take the derivative of each of the internal functions and package them into a new vector function.
\end{itemize}

\section{Arc Length}
\subsection{One-Variable Functions}
$$L = \int_a^b \sqrt{1+[f'(x)]^2} dx$$

\subsection{Parametric Functions}
Let $y(t)$ and $x(t)$ describe a parametric function in 2 dimensions. Then the arc length would be:
$$L = \int_a^b \sqrt{[x'(t)]^2 + [y'(t)]^2} dx$$

\subsection{Vector Funtions}
Let $\vec{r}(t)$ describe a vector function that converts a scalar $t$ into a vector. Then the arclength function would be:
$$L = \int_a^b |\vec{r}(t)| dt$$

\subsection{Reparamerizing with respect to Arc Length}
\paragraph{What is this? } Let there be a vector function $\vec{r}(t)$ and its correponding arc length function 
$s(t)$. Since $s$ is strictly increasing, we can safely \textbf{reparameterize} $\vec{r}(t)$ to be 
$\vec{r}(s(t)) \to \vec{r}(s)$. 
\paragraph{Why would you want to do this? } This type of reparameterization is useful because now we do not have to
rely on any particular coordinate system.
\paragraph{Steps to Reparameterizing}
\begin{enumerate}
\item Find $s(t) = \int_a^b |\vec{r}(u)| du$.
\item Put $s$ in terms of $t$.
\item Substitute the expression found in part 2 in the original $\vec{r}(t)$.
\end{enumerate}


\section{Partial Derivatives}
\subsection{Functions of Several Variables}
\paragraph{A function of two variables } transforms each pair of Reals $(x, y)$ in a given set to a single real number. The given set is the domain, and the set of reals that the pair is transformed to is the range.

\paragraph{Level functions } are functions that have $f(x, y) = k$ for given ranges of $(x, y)$

\paragraph{Functions of 3 or more variables } are pretty easy to extrapolate from functions of two variables, tbh.

\subsection{Limits and Continuity with Functions of Several Variables}
\subsubsection{Limits}
\paragraph{Definition of limit } with many variables:
$$\lim_{(x, y) \to (a, b)} f(x, y) = L$$
if for every number $\epsilon > 0$ there is a corresponding number $\delta > 0$ s.t.
if $0 < \sqrt{(x-a)^2 + (y-b)^2} < \delta$ then $|f(x, y) - L | < \epsilon$

\paragraph{How to find: }
Regard the non-mentioned variable in the notation as a constant and differentiate with respect to the mentioned variable.

\subsection{Higher Partial Derivatives}
$$(f_x)_y = f_{xy} = f_{12} = \frac{\partial}{\partial y} (\frac{\partial f}{\partial x}) = \frac{\partial ^2 f}{\partial y \partial x} = \frac{\partial z}{\partial y \partial x}$$

\subsubsection{Clairaut's Theorem}
If $f_{xy}$ and $f_{yx}$ are both \textbf{defined} and \textbf{continuous} on disk $D$ then:
$$f_{xy}(a, b) = f_{yx}(a, b)$$

\section{Gradient}
Think of the gradient like an operator that applies to functions of many variables (functions of vectors). The $\nabla$
just calculates the partial derivatative of the function with respect to each of its input variables and puts it
into a vector.

$$\nabla f(x, y) = [\frac{\partial f}{\partial x}, \frac{\partial f}{\partial y}]$$

Or, more generally for a function $f(\vec{x})$, $$\nabla f(\vec{x}) = [\frac{\partial f}{\partial x_1}, ... , 
\frac{\partial f}{\partial x_n}]$$

\section{Chain Rule with Many Variables}
Let there be a function $f(\vec{x})$. Let $\vec{x}$ of length $n$ be a function of $\vec{t}$ of length $m$. 
We take the partial derivative of $f$  with respect to $t_i$ by the following:

$$\frac{\partial f}{\partial t_i} = \frac{\partial f}{\partial x_1} \frac{\partial x_1}{\partial t_i} + 
... + \frac{\partial f}{\partial x_n} \frac{\partial x_n}{\partial t_i} $$

\chapter{Multiple Integrals}
Pretty much the same as regular integrals, you just do two. You can apply them to volumes under surfaces.

\section{Basic Meaning and Solving}

\section{Leibniz Integral Rule (Differentiability of Integral with Respect to Parameter)}
The Leibniz integral rule simply lets you more easily take the derivative of the integral of a multivariable
function where the variable you are integrating with respect to is not the same as the variable you are taking
the derivative with respect to.
$$\frac{d}{dx} \int_a^b f(x, t) dt$$

\subsection{Constant Bounds of Integration}
When you are integrating from one constant to another $[a, b]$, the result is quite simple and elegant.
$$\frac{d}{dx} \int_a^b f(x, t) dt = \int_a^b \frac{\partial f}{\partial x} dt$$

\subsection{Derivation}
Let's write $\frac{d}{dx} \int_a^b f(x, t) dt$ in terms of the definition of the derivative:
\begin{align*}
& = \frac{\int_a^b f(x +\Delta x, t)dt - \int_a^b f(x+\Delta x, t)dt}{\Delta x} \\
& = \frac{\int_a^b f(x, t) dx + \int_a^b \frac{\partial f}{\partial x} dt \Delta t dx - \int_a^b f(x, t) dt}{\Delta x} \\
& = \int_a^b \frac{\partial f}{\partial t} dx \\
\label{Constant Bounds}
\end{align*}

\subsection{Variable Bounds of Integration}
Final result:
$$\frac{d}{dx} \int_{a(t)}^{b^t} f(x, t) dt = \int_{a(t)}^{b(t)} \frac{\partial f}{\partial x} dt - f(a, t)\frac{da}{dt} + f(b, t)\frac{db}{dt}$$

\section{Polar Coordinates with Multiple Integrals}
Recall $r^2 = x^2 + y^2$, $x = rcos(\theta)$, $y = rsin(\theta)$

\paragraph{A polar rectangle} is of the form $$R = {(r, \theta) | a \leq r \leq b, \alpha \leq \theta \leq \beta}$$

\paragraph{Basic form of double integral in polar coordinates: }
$$\iint_R g dA = \int_{\alpha}^{\beta} \int_a^b g(r, \theta)\,dr\,d\theta$$

\subsection{Change to Polar Coordinates in Double Integrals}
$$\iint_R f(x, y) dA = \int_{\alpha}^{\beta}\int_a^b f(rcos(\theta), rsin(\theta))*r \,dr\,d\theta$$
Make sure not to forget the $r$ in the integral!

\subsection{Variable Bounds of Integration for $r$}
$$D = {(r, \theta) | \alpha \leq \theta \leq \beta, h(\theta) \leq r \leq g(\theta)}$$
Then:
$$\iint_D f(r, \theta) = \int_{\alpha}^{\beta} \int_{h(\theta)}^{h(\theta)} f(r, \theta)*r \, dr \, d\theta$$

\section{Applications of Multiple Integrals}
The obvious application of multiple integrals is to compute volume. But, like with regular 
integrals, there are a lot more ways you can apply them (think back to centre of mass, 
arc length, etc). 

\subsection{Density and Mass}
\paragraph{Lamina } is a plan of mass and surface density. With double integrals, 
we can now calculate mass of a lamina with \textbf{variable} density!

$$\rho(x, y) = \lim \frac{\Delta m}{\Delta A}$$
$$\rho(x, y) = \frac{\partial m}{dx dy}$$

\paragraph{Calculating Mass: }
$$m = \iint_D \rho(x, y) dA$$

\subsection{Moments of Intertia}

\paragraph{Moment $M_x$} is the distance of the centre of mass from the $x$-axis times 
the mass. Likewise with $y$.

\paragraph{Centre of Mass } is at $(\bar{x}, \bar{y})$ where $m\bar{x} = M_y$ and 
$m\bar{y} = M_x$.

If you're confused as to why the $\bar{x}$ and $M_y$ are related, just remember that 
the moment about the $y$-axis ($M_y$) is defined as the \textbf{distance from the 
y-axis to the centre of mass} times the mass of the lamina. That distance is the $\bar{x}$ 
value!

\subsection{Moments of Intertia}

Recall that $I = mr^2$ for particles.

$$I_x = \iint_D y^2 \rho(x, y)dA$$
$$I_y = \iint_D x^2 \rho(x, y)dA$$

$$I_o = I_x + I_y = \iint_D (x^2 + y^2) \rho(x, y)dA$$
Makes sense because $x^2+y^2$ is the distance of the chunk from the 
origin. 

\subsubsection{Radius of Gyration}
\paragraph{About Arbitrary Axis: } $R$ is the radius of gyration if $mR^2 = I$. Basically, if mass were concentrated at $R$,
then it would have the same moment of inertia as the original lamina.

\paragraph{About the Origin: } $(\bar{\bar{x}}, \bar{\bar{y}})$ is the point of gyration about the origin.

\subsection{Probability}
The probability function of a random variable $x$ is: 
$$f(x) > 0\,\forall \, x, \int_{-\infty}^{\infty} f(x) dx = 1$$

\paragraph{Probability of $a < x < b$: } $\int_a^b f(x) dx$

\paragraph{2-D Probability Function: } Chance of $(x, y) \in D$ is $\iint_D f(x, y) dA$.

Note that $\iint_{\mathbb{R}^2} f(x, y) dA = 1$

\subsubsection{Expected Values}
For single variable $x$ with probability density function $f$, the expected value is the 
average value of $x$ across many trials. 

$$\mu = \int_{-\infty}^{\infty} x f(x) dx$$

Makes sense because you're just summing the products of the chance of a particular $x$ 
and the value of that $x$ for all $x$.

\paragraph{Multi-Variable Functions: }
$$\mu_1 = X_{mean} = \iint_{\mathbb{R}^2} x f(x, y) dA$$
$$\mu_2 = Y_{mean} = \iint_{\mathbb{R}^2} y f(x, y) dA$$

\paragraph{Normal Distribution: } $$f(x) = \frac{1}{\sigma \sqrt{2\pi}} e^{-(x-\mu )^2/(2\sigma^2)}$$

\section{Surface Area}

\paragraph{How it's derived: } If you divide the surface in to an $m$ by $n$ grid, the 
surface area of each piece is the area of that parallelogram. That can be found by 
the cross product of the two vectors that define two sides of the parallelogram. Those 
vectors are just the width and length of the piece times the partial derivatives along 
that particular axis. 

$$A(S) = \iint_D \sqrt{(\frac{\partial f}{\partial x})^2 
                     + (\frac{\partial f}{\partial y})^2 + 1} dA$$

$$A(S) = \iint_D \sqrt{(\nabla f(x, y))^2 + 1} dA$$

The $+ 1$ term is there because it's $\frac{\partial f}{\partial f}$.

\section{Triple Integrals}
\subsection{Type 1}
Lies between continuous functions of $f(x, y)$

$$E = {(x, y, z) | (x, y) \in D, u_1(x, y) \leq z \leq u_2(x, y) }$$

Then $$\iiint_E f(x, y, z) dV = \iint_D [\int_{u_1(x, y)}^{u_2(x, y)} f(x, y, z) dz] dA$$

\subsection{Type 2}
Lies between continuous functions of $f(y, z)$

$$E = {(x, y, z) | (y, z) \in D, u_1(y, z) \leq x \leq u_2(y, z) }$$

Then $$\iiint_E f(x, y, z) dV = \iint_D [\int_{u_1(y, z)}^{u_2(y, z)} f(x, y, z) dx] dA$$


\subsection{Type 3}
Lies between continuous functions of $f(x, z)$
$$E = {(x, y, z) | (x, z) \in D, u_1(x, z) \leq y \leq u_2(x, z) }$$

Then $$\iiint_E f(x, y, z) dV = \iint_D [\int_{u_1(x, z)}^{u_2(x, z)} f(x, y, z) dy] dA$$


\subsection{Applications of Triple Integrals}
\subsubsection{Obvious}
$$V = \iiint_E dV$$
$$m = \iiint_E \rho(x, y, z) dV$$

\subsubsection{Moments and CoM}
\paragraph{REMEMBER: } Moments are about a plane (e.g. $(x, y)$ plane) while moments of INERTIA are 
about an axis (think about spinning vs. balancing - you balance along a plane, spin around an axis).

$$M_{yz} = \iiint_E x\rho(x, y, z) dV$$
etc.

$$I_x = \iiint_E (y^2 + z^2) \rho(x,y,z)dV$$
\end{document}

