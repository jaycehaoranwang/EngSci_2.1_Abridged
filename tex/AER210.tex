\documentclass[a4paper,12pt]{report}

\usepackage{amsmath,amsfonts,mathtools}
\usepackage{hyperref}

\begin{document}
\title{AER210 Abridged}
\author{Aman Bhargava}
\date{September 2019}
\maketitle

\tableofcontents

\pagebreak

\section{Review: Stuff to Have Memorized}
%TODO: Populate this with stuff from MAT195

\chapter{In-Class Review}
\section{Vectors \& Vector Functions}

\begin{itemize}
\item Vector = magnitude + direction
\item If the origin of the vector is the origin of the coordinate system, it's a position vector.
\item Dot product: 
$\vec{a} \cdot \vec{b} = \vec{a}_1 \cdot \vec{b}_1 + ... + \vec{a}_n \cdot \vec{b}_n$
\item Cross product: $\vec{a} \times \vec{b} = det(i, j, k; \vec{a}^{T}; \vec{b}^{T})$
\item Cross product is the area of the paralellogram traced out by the two vectors.
\item Scalar triple product: $\vec{a} \cdot (\vec{b} \times \vec{c})$, produces a scalar, represents the volume of the parallelepiped formed by the three vectors.
\item To get the derivative of a vector function, simply  take the derivative of each of the internal functions and package them into a new vector function.
\end{itemize}

\section{Arc Length}
\subsection{One-Variable Functions}
$$L = \int_a^b \sqrt{1+[f'(x)]^2} dx$$

\subsection{Parametric Functions}
Let $y(t)$ and $x(t)$ describe a parametric function in 2 dimensions. Then the arc length would be:
$$L = \int_a^b \sqrt{[x'(t)]^2 + [y'(t)]^2} dx$$

\subsection{Vector Funtions}
Let $\vec{r}(t)$ describe a vector function that converts a scalar $t$ into a vector. Then the arclength function would be:
$$L = \int_a^b |\vec{r}(t)| dt$$

\subsection{Reparamerizing with respect to Arc Length}
\paragraph{What is this? } Let there be a vector function $\vec{r}(t)$ and its correponding arc length function 
$s(t)$. Since $s$ is strictly increasing, we can safely \textbf{reparameterize} $\vec{r}(t)$ to be 
$\vec{r}(s(t)) \to \vec{r}(s)$. 
\paragraph{Why would you want to do this? } This type of reparameterization is useful because now we do not have to
rely on any particular coordinate system.
\paragraph{Steps to Reparameterizing}
\begin{enumerate}
\item Find $s(t) = \int_a^b |\vec{r}(u)| du$.
\item Put $s$ in terms of $t$.
\item Substitute the expression found in part 2 in the original $\vec{r}(t)$.
\end{enumerate}


\section{Partial Derivatives}
\subsection{Functions of Several Variables}
\paragraph{A function of two variables } transforms each pair of Reals $(x, y)$ in a given set to a single real number. The given set is the domain, and the set of reals that the pair is transformed to is the range.

\paragraph{Level functions } are functions that have $f(x, y) = k$ for given ranges of $(x, y)$

\paragraph{Functions of 3 or more variables } are pretty easy to extrapolate from functions of two variables, tbh.

\subsection{Limits and Continuity with Functions of Several Variables}
\subsubsection{Limits}
\paragraph{Definition of limit } with many variables:
$$\lim_{(x, y) \to (a, b)} f(x, y) = L$$
if for every number $\epsilon > 0$ there is a corresponding number $\delta > 0$ s.t.
if $0 < \sqrt{(x-a)^2 + (y-b)^2} < \delta$ then $|f(x, y) - L | < \epsilon$

\paragraph{How to find: }
Regard the non-mentioned variable in the notation as a constant and differentiate with respect to the mentioned variable.

\subsection{Higher Partial Derivatives}
$$(f_x)_y = f_{xy} = f_{12} = \frac{\partial}{\partial y} (\frac{\partial f}{\partial x}) = \frac{\partial ^2 f}{\partial y \partial x} = \frac{\partial z}{\partial y \partial x}$$

\subsubsection{Clairaut's Theorem}
If $f_{xy}$ and $f_{yx}$ are both \textbf{defined} and \textbf{continuous} on disk $D$ then:
$$f_{xy}(a, b) = f_{yx}(a, b)$$

\section{Gradient}
Think of the gradient like an operator that applies to functions of many variables (functions of vectors). The $\nabla$
just calculates the partial derivatative of the function with respect to each of its input variables and puts it
into a vector.

$$\nabla f(x, y) = [\frac{\partial f}{\partial x}, \frac{\partial f}{\partial y}]$$

Or, more generally for a function $f(\vec{x})$, $$\nabla f(\vec{x}) = [\frac{\partial f}{\partial x_1}, ... , 
\frac{\partial f}{\partial x_n}]$$

\section{Chain Rule with Many Variables}
Let there be a function $f(\vec{x})$. Let $\vec{x}$ of length $n$ be a function of $\vec{t}$ of length $m$. 
We take the partial derivative of $f$  with respect to $t_i$ by the following:

$$\frac{\partial f}{\partial t_i} = \frac{\partial f}{\partial x_1} \frac{\partial x_1}{\partial t_i} + 
... + \frac{\partial f}{\partial x_n} \frac{\partial x_n}{\partial t_i} $$

\chapter{Multiple Integrals}



\end{document}
