\documentclass[a4paper,12pt]{report}

\usepackage{amsmath,amsfonts,mathtools}
\usepackage{hyperref}

\begin{document}
\title{MAT292 Abridged}
\author{Aman Bhargava}
\date{September 2019}
\maketitle

\tableofcontents

\section{Introduction}
The textbook and lectures for this course offer a great comprehensive guide for the 
methods of solving ODE's. The goal here is to give a very concise overview of the things you 
need to know (NTK) to answer exam questions. Unlike
some of our other courses, you don't need to be very intimately familiar with the derivations
of everything in order to solve the problems (though it certainly doesn't hurt). Think of this 
as a really good cheat sheet.

\chapter{Qualitative Things and Definitions}
\section{Definitions}
\begin{enumerate}
\item Order:
\item Autonomous:
\end{enumerate}


\section{Qualitative Analytic Methods to Know}
\begin{enumerate}
\item Phase lines
\item Slope fields
\end{enumerate}

\section{Types of Equilibrium}
\begin{enumerate}
\item Asymptotic stable equilibrium 
\item Unstable equilibrium 
\item Semistable equilibirium
\end{enumerate}

\chapter{1st Order ODE's}
\section{Separable 1st Order ODE's}
If you can write the ODE as: $$\frac{dy}{dx} = p(x)q(y)$$

Then you can put $p(x)$ with $dx$ on one side and $q(y)$ with  $dy$ on the other and 
integrate them both so solve the ODE.

\section{Method of Integrating Factors}
This is used to solve ODE's that can be put into the form 
$$\frac{dy}{dt} + p(t)*y = g(t)$$

The chain rule can be written as: $\int (f'(x)g(x) + f(x)g'(x)) dx =  f(x)g(x)$

We can use an \textbf{integrating factor} equivalent to $e^{\int p(t) dt}$ to multiply
both sides and arrive at a form that can be integrated with ease using the reverse chain 
rule.

\section{Exact Equations}
If the equation is of the form $$M(x, y) + N(x, y) \frac{dy}{dx} = 0$$
and $$M_y(x, y) = N_x(x, y)$$ 
then $\exists$ a function $f$ satistfying $$f_x(x, y) = M(x, y); f_y(x, y) = N(x, y)$$

\paragraph{The solution: } $f(x, y) = C$ where $C$ is an arbitrary constant.

\chapter{Systems of Two 1st Order DE's}




\end{document}