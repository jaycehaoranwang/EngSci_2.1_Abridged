\documentclass[a4paper,12pt]{report}

\usepackage{amsmath,amsfonts,mathtools}
\usepackage{hyperref}

\begin{document}
\title{PHY293 Abridged}
\author{Aman Bhargava}
\date{September 2019}
\maketitle

\tableofcontents

\section{Introduction}
Prof. Grisouard has put up some pretty solid 
\href{https://github.com/PHY293-Grisouard/Chapters-2019}{notes} written in jupyter notebooks. 
Those offer a great comprehensive guide for the course contents. The goal here is to give
a very concise overview of the things you need to know (NTK) to answer exam questions. Unlike
some of our other courses, you don't need to be very intimately familiar with the derivations
of everything in order to solve the problems (though it certainly doesn't hurt). Think of this 
as a really good cheat sheet.

\chapter{Simple Harmonic Oscillators (SHO's)}
\section{Set Up}
Pretty much the same as covered last year.
$$F_{restorative} = -kx$$
$$F_{net} = ma$$
$$ma + kx = 0$$
We let $\omega_0^2 = k/m$ and rename $a$ to $\ddot{x}$
\section{ODE and Solution}
\paragraph{ODE:}
$$\ddot{x} + \omega_0^2 x = 0$$
\paragraph{Solution:}
$$x(t) = a cos(\omega_0 t + \theta)$$
Solve for 2 unknowns (usually $a$ and $\theta$) based on two initial conditions.

\chapter{Dampened Harmonic Oscillators (DHO's)}
\section{Set Up}
We have all the same forces from last time plus a force of friction. We kind of have our
hands tied with what models of friction we can use (remember how friction can be proportional
to velocity, acceleration?) because this course only deals with linear systems. Therefore our
one and only formula for friction is:

$$F_{friction} = F_f = -bv$$

\section{ODE and Solutions}
\paragraph{ODE}
We let $\gamma = b/m$ so that:
$$\ddot{x} + \gamma \dot{x} + \omega_0^2 x = 0$$

\subsection{General Solution}

And for underdampened and over dampened, the overall solution looks like:
$$x(t) = ae^{rt}$$
Where $a, r \in \C$, remembering that $$e^{i\theta} = i sin(\theta) + cos(\theta)$$

This is actually really cool, because the existence of a complex component in $r$ is what 
enables oscillation. 

If we populate the initial ODE with the derivatives of our $x(t)$, we get:
$$ar^2e^{rt} + \gamma are^{rt} + \omega ae^rt = 0$$
$$r^2 + \gamma r + \omega = 0$$

This results in a simple quadratic for which you can solve for $r$ using the quadratic formula.

$$x = \frac{b \pm \sqrt{b^2 - 4ac}}{2a}$$

There will be up to 2 $r$ values. Since the solution space is a 2-D vector space, any 
linear combination of vectors (i.e. values of $r$) is still a valid vector.

You might be wondering how $a$ figures into this whole thing. That's just based on initial
conditions, really. 

\subsection{Regieme 1: Underdampened}
If the dampening is weak enough, there will still be \textbf{some} oscillation before the 
oscillator comes to rest. In this case, the general solution is:

$$x(t) = A_0 e^{-\gamma t/2} cos(w_dt + \phi)$$
Where $w_d^2 = w_dampened^2 = \omega_0^2 - \frac{\gamma^2}{4}$

You get this by solving for $r$ with the quadratic formula, plugging the two imaginary 
roots in, then taking just the real components because you know that $x(t)$ is real.

\subsubsection{Logarithmic Decrement}
This quantity is the $ln$ of the ratio of 1 peak to the next.

$$\frac{\gamma T_d}{2}$$

Where $T_d$ is the period based on $w_d$.

\subsection{Regieme 2: Heavily Dampened}
In this case both roots are real. This occurs when $\gamma^2-4\omega_0^2 > 0$

Let $r_p, r_m$ be the two real roots. Then the solution is:

$$x(t) = a_pe^{r_pt} + a_me^{r_mt}$$

You can solve for the a values at leisure with two initial conditions.

\subsection{Regieme 3: Critically Dampened}
When the discriminant of the quadratic is zero (i.e. $\gamma^2 = \4\omega_0^2 = $), you
get an overconstrained problem because you have fewer unknowns than you have constraints.
Therefore, the new solution for this exact case is:

$$x(t) = (A + Bt)e^{-\gamma t/2}$$

And it decays the fastest and is pretty cool. Yay.

\section{Energy}
$$E = K + U$$
$$E = \frac{1}{2}mv^2 + \frac{1}{2}kx^2$$
$$\frac{dE}{dt} = -bv^2$$

\subsection{Underdampened Energy}
The equations get really messy so we assume that $\omega_0 >> \gamma/2$ and $\omega_0 = \omega_d$. After 
you sub everything in and use the pythagorean identity to cancel out the squared sine/cosine,
you get:

$$E = \frac{1}{2} kA_0^2e^{-\gamma/t}$$

\subsection{Critical and Overdampened Energy}
These ones are really easy, they just decay and that's it. Just remember the equations for 
energy and solve for the velocity equation.

\section{Q-Factor}
$Q$ is a measure of the tendency to oscillate dividied by the tendency to dampen.
$$Q = frac{\omega_0}{\gamma}$$
Let $\tau = \frac{1}{gamma}$ be proportional to the lifetime of the oscillator. Then 
$n = \frac{\tau}{T_0}$ is the number of cycles in a lifetime.
$$Q = 2\pi n$$


$Q$ is also proportional to the rate at which dampening removes energy from the system. Then
$$E_n = E_0e^{-\gamma t}$$
Represents the amount of mechanical energy in the system at time $t$.
$$\frac{E_n - E_{n+1}}{E_n} = ... = \frac{2\pi}{Q}$$

Therefore $Q$ is the energy in oscillator divided by the amount of energy lost at the
following \textbf{radian}.




\end{document}