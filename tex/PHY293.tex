\documentclass[a4paper,12pt]{report}

\usepackage{amsmath,amsfonts,mathtools}
\usepackage{amssymb}
\usepackage{hyperref}

\begin{document}
\title{PHY293 Abridged}
\author{Aman Bhargava}
\date{September 2019}
\maketitle

\tableofcontents

\section{Introduction}
Prof. Grisouard has put up some pretty solid 
\href{https://github.com/PHY293-Grisouard/Chapters-2019}{notes} written in jupyter notebooks. 
Those offer a great comprehensive guide for the course contents. The goal here is to give
a very concise overview of the things you need to know (NTK) to answer exam questions. Unlike
some of our other courses, you don't need to be very intimately familiar with the derivations
of everything in order to solve the problems (though it certainly doesn't hurt). Think of this 
as a really good cheat sheet.

\chapter{Simple Harmonic Oscillators (SHO's)}
\section{Set Up}
Pretty much the same as covered last year.
$$F_{restorative} = -kx$$
$$F_{net} = ma$$
$$ma + kx = 0$$
We let $\omega_0^2 = k/m$ and rename $a$ to $\ddot{x}$
\section{ODE and Solution}
\paragraph{ODE:}
$$\ddot{x} + \omega_0^2 x = 0$$
\paragraph{Solution:}
$$x(t) = a cos(\omega_0 t + \theta)$$
Solve for 2 unknowns (usually $a$ and $\theta$) based on two initial conditions.

\chapter{Dampened Harmonic Oscillators (DHO's)}
\section{Set Up}
We have all the same forces from last time plus the force of friction.

$$F_{friction} = F_f = -bv$$

\section{ODE and Solutions}
\paragraph{ODE}
We let $\gamma = b/m$ so that:
$$\ddot{x} + \gamma \dot{x} + \omega_0^2 x = 0$$

\subsection{General Solution}

And for underdampened and over dampened, the overall solution looks like:
$$x(t) = ae^{rt}$$
Where $a, r \in \mathbb{C}$, remembering that $$e^{i \theta} = i sin(\theta) + cos(\theta)$$

This is actually really cool, because the existence of a complex component in $r$ is what 
enables oscillation. 

If we populate the initial ODE with the derivatives of our $x(t)$, we get:
$$ar^2e^{rt} + \gamma are^{rt} + \omega_0^2 ae^rt = 0$$
$$r^2 + \gamma r + \omega_0^2 = 0$$

This results in a simple quadratic for which you can solve for $r$ using the quadratic formula.

$$x = \frac{b \pm \sqrt{b^2 - 4ac}}{2a}$$
$$r = \frac{\gamma \pm \sqrt{\gamma^2 - 4w_0^2}}{2}$$

There will be up to 2 $r$ values. Since the solution space is a 2-D vector space, any 
linear combination of vectors (i.e. values of $r$) is still a valid vector.

You might be wondering how $a$ figures into this whole thing. That's just based on initial
conditions, really. 

\subsection{Regieme 1: Underdampened}
If the dampening is weak enough, there will still be \textbf{some} oscillation before the 
oscillator comes to rest. In this case, the general solution is:

$$x(t) = A_0 e^{-\gamma t/2} cos(w_dt + \phi)$$
Where $w_d^2 = w_{dampened}^2 = \omega_0^2 - \frac{\gamma^2}{4}$

You get this by solving for $r$ with the quadratic formula, plugging the two imaginary 
roots in, then taking just the real components because you know that $x(t)$ is real.

\subsubsection{Logarithmic Decrement}
This quantity is the $ln$ of the ratio of 1 peak to the next.

$$\frac{\gamma T_d}{2}$$

Where $T_d$ is the period based on $w_d$.


\subsection{Regieme 2: Heavily Dampened}
In this case both roots are real. This occurs when $\gamma^2-4\omega_0^2 > 0$

Let $r_p, r_m$ be the two real roots. Then the solution is:

$$x(t) = a_pe^{r_pt} + a_me^{r_mt}$$

You can solve for the a values at leisure with two initial conditions.

\subsection{Regieme 3: Critically Dampened}
When the discriminant of the quadratic is zero (i.e. $\gamma^2 = 4\omega_0^2 = $), you
get an overconstrained problem because you have fewer unknowns than you have constraints.
Therefore, the new solution for this exact case is:

$$x(t) = (A + Bt)e^{-\gamma t/2}$$

And it decays the fastest and is pretty cool. Yay.

\section{Energy}
$$E = K + U$$
$$E = \frac{1}{2}mv^2 + \frac{1}{2}kx^2$$
$$\frac{dE}{dt} = -bv^2$$

\subsection{Underdampened Energy}
The equations get really messy so we assume that $\omega_0 >> \gamma/2$ and $\omega_0 = \omega_d$. After 
you sub everything in and use the pythagorean identity to cancel out the squared sine/cosine,
you get:

$$E = \frac{1}{2} kA_0^2e^{-\gamma t}$$

\subsection{Critical and Overdampened Energy}
These ones are really easy, they just decay and that's it. Just remember the equations for 
energy and solve for the velocity equation.

\section{Q-Factor}
$Q$ is a measure of the tendency to oscillate dividied by the tendency to dampen.
$$Q = \frac{\omega_0}{\gamma}$$
Let $\tau = \frac{1}{gamma}$ be proportional to the lifetime of the oscillator. Then 
$n = \frac{\tau}{T_0}$ is the number of cycles in a lifetime.
$$Q = 2\pi n$$


$Q$ is also proportional to the rate at which dampening removes energy from the system. Then
$$E_n = E_0e^{-\gamma t}$$
Represents the amount of mechanical energy in the system at time $t$.
$$\frac{E_n - E_{n+1}}{E_n} = ... = \frac{2\pi}{Q}$$

Therefore $Q$ is the energy in oscillator divided by the amount of energy lost at the
following \textbf{radian}.

\chapter{Forced Oscillators}
\section{Equation of Motion ODE}
$$\ddot{x} + \omega_0^2 x = F_0' cos(\omega t)$$
Where $\omega_0^2 = k/m$, $F_0' = F_0/m$, $\omega$ is the driving frequency.

\section{Solution: Undamped Forced Oscillator}
$$x(t) = A(\omega) cos(\omega t - \delta(\omega))$$
Where
$$\delta(\omega_0) = 0 \,\,if\,\, \omega < \omega_0, else \pi$$
If you shake it too fast, it'll vibrate in exactly opposite phase. 
$$A(w) = |\frac{\omega_0^2}{\omega_0^2 - \omega^2}A_f|$$

\subsection{$\omega = \omega_0$}
Perfect resonance makes amplitude tend toward infinity.
$$x(t) = \frac{A_f \omega_0}{\sqrt{2}} t cos(\omega_0t - \frac{pi}{4})$$

\section{Solution: Forced Damped Oscillators}
Equation of Motion: $\ddot{x} + \gamma \dot{x} + \omega_0^2 x = A_f w_0^2 cos(wt)$

\paragraph{General Solution: } Same as before 
$$x(t) = A(\omega) cos(\omega t - \delta(\omega))$$

\paragraph{Updates to $A$ and $\delta(\omega)$: } 
$$A(\omega) = \frac{\omega_0^2}{\sqrt{\gamma^2\omega^2+(\omega_0^2 - \omega)^2}}A_f$$
$$tan[\delta(\omega)] = \frac{\gamma \omega}{\omega_0^2-\omega^2}$$

\paragraph{Maximum $A(w)$} occurs when $\omega = \omega_0^2 - \frac{\gamma^2}{2}$. In other words, 
$$\omega_{max} = \omega_0^2[1-\frac{1}{2Q^2}]$$

$$\therefore A_{max} = \frac{Q}{\sqrt{1-\frac{1}{4Q^2}}}A_f$$

\section{Transient Response}
When you suddenly start a damped, forced oscillator, you have one mode that is the long-term response and one 
that is the transient response (as if there is no forcing).

$$x_{free} = A_{free} e^{-\gamma t / 2}cos(\omega_d t + \phi)$$
$$x_{forced} = A(\omega) cos(\omega t - \delta(\omega))$$

$x_{free}$ tend toward zero and only exists for the beginning, so it is the transient response. 

\section{Power in FDHO's}
$$\therefore -\dot{E}(w, t) = P(w, t) = bv^2 = bV^2(w)sin^2(wt-\delta)$$
Where $V$ is the amplitude of the velocity.

\subsection{Average Power Per Cycle}
$$\bar{P}(w) = \frac{F_0^2}{2m} \frac{\gamma}{\gamma^2 + (\omega_0^2/\omega - \omega)^2}$$
\paragraph{Maximizing Case: } $\bar{P}$ is maximized when $w = w_0$

\subsection{Power vs. $\omega$}
Let us plot $\bar{P}(w)$ vs. $\omega / \omega_0$. The $\omega_{full width half height} = \omega_{fwhh} = \gamma$

\chapter{Coupled Oscillators}
\paragraph{Main Takeaways}
\begin{itemize}
\item Motion of coupled systems is a linear combination of orthogonal forms of motion.
\item Motion can be found via 2x2 matrix system
\item Each normal mode has a unique frequency common to all elements. 
\item `Eigen' = `normal'.
\item \textbf{Beating} = when frequencies of modes are very close.
\end{itemize}

\subsection{How to Solve Coupled Systems}
From tutorial note: 
\subsubsection{Determine Equations of Motion}
You must give each mass its own coordinate system where deviation in one direction adds to the zero 
coordinate and deviation to the other subtracts from the zero coordinate.

Implement $f = ma$ and $f = kx$ to create equations of motion of each mass depending on its position 
and the position of its neighbouring masses.

Now we have $$m\ddot{x_a} = ...$$ $$...$$ $$m\ddot{x_c} = ...$$

\subsubsection{Assume $\exists$ a normal mode}
This implies that $$x_a = A cos(wt)$$ $$...$$ $$x_c = C cos(wt)$$

All we need to solve for now is $w$, and $A \to C$. To do so, we sub our definitions for $x_a, x_b, and x_c$ 
into our initial EoM, which gives us relationships between $A, B, C, w$ (cosines cancel out!)

Now we just have to put everything in a matrix equation of the form:
$$[...][A, B, C] = [0, 0, 0]$$
$$[M][A, B, C] = [0, 0, 0]$$

which implies that $det(M) = 0$. This will give us $3$ different eigen frequencies (or however many degrees 
of freedom we have). From there, we just have to solve for the coefficients $A, B, C$ via initial conditions, 
remembering that $x_a = Acos(wt)$ and so on!

\subsection{Review: Results from Linear Algebra}
\begin{itemize}
\item $Av = \lambda v$, $(A-\lambda I)v = 0$
\item We get non-trivial solutions for $v$ if $det(A - \lambda I) = 0$
\item A matrix is diagonizable if $\exists n$ eigen values $\lambda$ (sufficient, not necessary)
\item Each solution to the initial $Ax = b$ is unique because the eigen vectors form a basis.
\end{itemize}

\paragraph{Connecting Eigen Things to Real Things}
\begin{itemize}
\item Each $\lambda_i = \omega_i^2$
\end{itemize}

\chapter{Standing Waves}
\paragraph{Traveling wave: } a traveling disturbance where no matter travels on average.
\paragraph{Continuum: } infinite coupled oscillators $\to$ infinite modes can arise, characterized 
by nodes and antinodes.

Boundaries give rise to standing waves (think about boundaries on a plucked guitar string).

\section{The Taut String}
Properties of a taut string:
\begin{itemize}
\item $tan\theta = \frac{dy}{dx} << 1$ where $x$ is along the string and $y$ is across the string.
\item Vibrations occur in only one plane
\item \textbf{Phase speed } is $v^2 = T/\mu$ with $T$ as \textbf{tension} and $\mu$ as 
\end{itemize}

\paragraph{WAVE EQUATION: } $$\frac{\partial^2 y}{\partial t^2} = v^2 \frac{\partial^2y}{\partial x^2}$$

\section{Solution to Wave Equation}
We assume that the solution is of the form $$y(x, t) = f(x)h(t)$$ so that it is seperable and works nicely 
with the wave equation. When we sub this into the wave equation, we get:

Let $$\frac{1}{h(t)} \frac{d^2h(t)}{dt^2} = -w^2 = v^2 \frac{1}{f(x)} \frac{d^^2f(x)}{dx^2}$$

So we can still have a $w^2$ that fits well into our SHO ODE:

$$h(t) = H cos(wt + \phi)$$ $$f(x) = A cos(kx) + B sin(kx)$$ where $k = w/v$

\subsection{Boundary Conditions}
We know that $y(t, 0) = y(t, L) = 0$ because the string is secured on either end. This forces the $A$ term 
in the equation $f(x) = A cos(kx) + B sin(kx)$ to be zero, so now $f(x) = B sin(kx)$. Now $f(kL) = 0$ for 
any natural number $k$. 

$$k_n = \frac{n \pi}{L} \,\,, \,\, n \in \mathbb{N}$$

Recall that $k = w/v$ so now $w_n = \frac{n \pi v}{L}$. 

\subsection{Final Solution}
$$y_n(x, t) = C_n cos(w_n t + \phi_n)sin(k_n x)$$
$$y(x, t) = \sum_{n=1}^{\infty} y_n(x, t)$$

\subsection{Surrounding Math}
We need some tools to do proofs and reasoning about these standing and traveling waves. 
$$f \odot g = \frac{2}{L} \int_0^L f(x)g(x) dx$$
which represents the inner product of the vector space and 
$\delta_{mn}$ which is 1 when $m=n$.
$$sin(k_n x) \odot sin(k_m x) = \delta_{mn}$$

Because they are orthogonal to eachother.

\subsection{Energy Considerations}
$$E_n = \frac{1}{4} \mu L w_n^2 A_n^2$$

\chapter{Traveling Waves}
The real general solution to the wave equation: 
$$\frac{\partial^2 y}{\partial t^2} = v^2 \frac{\partial^2 y}{\partial x^2}$$
$$y(x, t) = f(x-vt) + g(x+vt)$$

It can be shown via trig sums that this is the same as the standing wave solution when you 
put it between two boundaries.

\paragraph{Most common traveling wave function: } $$y(x, t) = A sin(\frac{2 \pi}{\lambda} (x \pm vt))$$
$$= A sin(k(x\pm vt))$$
$$= A sin(kx \pm wt)$$
\begin{itemize}
\item $\lambda$ = wavelength
\item $k$ = wavenumber
\item $w$ = angular frequency
\item $\frac{2\pi}{w}$ = period
\end{itemize}

\section{Energy Considerations}
$$K = \frac{1}{2} \mu (\frac{\partial y}{\partial t})^2$$
$$U = \frac{1}{2} \mu v^2 (\frac{\partial y}{\partial x})^2$$

$<>$ corresponds to integration over one wavelength, so $<f> = \int_0^{\lambda} f(x) dx$.

To find $<E>$ we can apply the above formulae for $K$ and $U$ to the general solution 
$y(x, t)=A cos(kx-wt)$.

$$<E> = \frac{1}{2} \mu \lambda A^2 w^2$$

\section{Power Considerations}
Since these waves are unbounded traveling entites, it's useful to consider how much 
energy they transport past a certain point per second.

We start with instantaneous energy $E = \mu A^2 w^2 sin^2(kx - wt)$. Then 
$$\bar{P} = \frac{w <E>}{2 \pi} = \frac{\mu \lambda A^2 w^3}{4 \pi} = \frac{\mu A^2 w^2 v}{2}$$

\section{Reflection and Transmission}
When medium changes discontinuously, you get reflection and transmission. The boundary conditions 
at the discontinuity are 
\paragraph{Boundary Condition 1: } matching frequency, amplitude, and phase.
\paragraph{Boundary Condition 2: } transverse (across-string) forces must match.

Therefore we know that 
$$y_T(x, t) = A_T cos(wt - k_t x)$$ $$y_R = A_R cos (wt + k_r x)$$

have the same $wt$ term, and $k_R, k_T > 0$ so they propagate in the right directions. The reflected 
wave is in the same original medium, so its spatial frequency $k_R = k_I = w/v_1$ while $k_T = w/v_2$.

Now the only numbers that aren't known are $A_R, A_T$. According to boundary condition 1, $A_I + A_R = A_T$. 
Now we use boundary condition to solve for 
$$\frac{A_T}{A_I} = \frac{2k_1}{k_1+k_2} = \pmb{\tau_{12}} = \frac{2 \sqrt{\mu_1}}{\sqrt{\mu_1} + \sqrt{\mu_2}} $$ 
$$\frac{A_R}{A_I} = \frac{k_1 - k_2}{k_1 + k_2} = \pmb{\rho_{12}} = \frac{\sqrt{\mu_1} - \sqrt{\mu_2}}{\sqrt{\mu_1} + \sqrt{\mu_2}}$$

\section{Traveling Waves in Multiple Dimensions}
...

\end{document}